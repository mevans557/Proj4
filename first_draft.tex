\documentclass[11pt]{article}
\usepackage{cite}

\begin{document}

\title{An Exploration of Disease Modelling on Random Graphs}
\author{Matthew Evans}
\date{03/11/2025}
\maketitle
\newpage

\tableofcontents
\newpage

\section{Abstract}

\paragraph{} Need to do this bit.

\paragraph{} TODO: Add an Abstract \cite{krauseAndrewKrauseDurham}


\section{Introduction}
\paragraph{} Modelling the spread of disease is an important topic in Epidemiology, often informing policy decisions and national healthcare decisions \cite{?}. Since the COVID-19 pandemic, modelling of disease spread has garnered more public interest, with major news organisations at one point frequently reporting on 'R-numbers' and other modelling parameters\cite{?}. The far-reaching impact of disease modelling can therefore not be understated, as even though global pandemics are infrequent, country or community-wide outbreaks are relatively frequent\cite{?}. Developing more complex and accurate models of disease spread therefore has the potential to change how governments and healthcare professionals respond to outbreaks of infectious disease\cite{?}, or how they manage diseases such as Malaria which are endemic to certain areas \cite{?}. Accurate modelling of disease spread also increases our understanding of various complex diseases, and can provide better preparedness for future outbreaks, both in containing diseases and managing how an outbreak is handled \cite{?}. For example, the famous "flatten the curve" campaign during COVID was backed up by research in disease modelling, and the slogan itself actually makes specific reference to the curves generated by SIR models \cite{?}.
\paragraph{} Disease modelling is frequently done using continuous-space compartmental models\cite{?}. These range in complexity from the basic SI model, up to much more varied models such as MSEIRS\cite{?}, or SIR models with social stress\cite{?}. These are generally done with an assumption of spatial homogeneity, or well-mixed populations\cite{?}. There has been some work in also adding spatial modelling\cite{?}, particularly with diffusion-based models\cite{?}. However, these still make an assumption of spatial continuity, which is not the case for an actual population\cite{?}. A small amount of work has been done with disease modelling on graph structures \cite{croccoloSpreadingInfectionsRandom2020}, \cite{browneInfectionPercolationDynamic2021}. However, these have generally been limited to more simplified models. There has also been some work done on graph structures to model disease spread between cities, however in these models the cities are still assumed to be spatially homogeneous\cite{?}.

\section{Comparison of Implementations}

\section{Building the Model}

\section{Analysis of Criticality}

\newpage
\section{Bibliography}
\raggedright
\bibliography{mybib}
\bibliographystyle{plain}
\end{document}