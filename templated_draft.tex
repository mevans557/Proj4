%%%%%%%%%%%%%%%%%%%%%
% Version 1 17/10/25


\documentclass[12pt,a4paper,openany,twoside]{book}
\usepackage[top=1in, bottom=1in, left=0.8in, right=0.8in]{geometry} % 0.8 inch margins

% Language and date format
\usepackage[UKenglish]{babel} 
\usepackage[UKenglish]{isodate}
\cleanlookdateon 

% Fonts
\usepackage[T1]{fontenc} %DO NOT CHANGE
\usepackage{lmodern} % better Latin Modern fonts %DO NOT CHANGE
\usepackage{microtype} % improves justification and kerning 

% Graphics and captions
\usepackage{graphicx} 
\usepackage[dvipsnames]{xcolor} % Gives you colours! 
\usepackage[margin=15pt,font=footnotesize,labelfont=bf,format=hang]{caption} 
\usepackage{subcaption} 
\usepackage{pgfplots} % you can import graphs as PGF files 
\pgfplotsset{compat=1.18} 
\usepackage{float}

% Maths
\usepackage{amsmath,amssymb,mathtools} 
\usepackage{bm} % Making letters bold 
\usepackage{isomath} % Modern standards 

% Theorems, definitions, etc.
\usepackage{amsthm} 
\newtheorem{theorem}{Theorem}[chapter] % Number theorem per chapter
\newtheorem{lemma}[theorem]{Lemma} % Number lemmas like theorems
\newtheorem{definition}{Definition}[chapter] % Number definitions per chapter

% Lists, tables, units
\usepackage{enumitem} % Allows for formatting of lists 
\setlist{topsep=0pt} 
\usepackage{booktabs} % Prettier tables 
\usepackage{siunitx} % Typeset measurements correctly 

% Referencing
\usepackage{natbib}

\bibliographystyle{abbrvnat} % possible citation style
%\bibliographystyle{plain} % possible citation style
%\setcitestyle{authoryear,open={(},close={)}} % Many options available. Can be changed after discussing with your supervisor.
\usepackage[pdfencoding=auto,pdftitle={Dissertation},pdflang={en-GB}]{hyperref} % Make links work
\hypersetup{
	colorlinks = true,
	linkcolor = black,
	citecolor = black
}
\usepackage{cleveref} % Easy internal referencing 

% Formatting
\usepackage[raggedright]{titlesec} %DO NOT CHANGE
\usepackage{fancyhdr} %DO NOT CHANGE

%% Subfiles so you can write separate chapter files [You can enable this if you're using subfiles]
%\usepackage{subfiles}
%\newcommand{\subfilebibliography}{%
%	\ifSubfilesClassLoaded{%
%		\bibliography{references.bib}
%	}{}
%} % Makes the bibliography work for every chapter file

% Headers and footers. 
\pagestyle{fancy} %DO NOT CHANGE
\renewcommand{\headrulewidth}{0pt} %DO NOT CHANGE
\renewcommand\sectionmark[1]{\markright{\thesection\; #1}} %DO NOT CHANGE
\fancyhead[RO]{\small\it\lowercase\nouppercase\selectfont\rightmark} %DO NOT CHANGE
\fancyhead[LO]{}%DO NOT CHANGE
\fancyhead[LE]{\small\it\lowercase\nouppercase\selectfont\rightmark} %DO NOT CHANGE
\fancyhead[RE]{} %DO NOT CHANGE
\setlength{\headheight}{15pt} %DO NOT CHANGE
\fancypagestyle{blank}{\fancyhf{}\fancyfoot[C]{}\renewcommand{\headrulewidth}{0pt}} %DO NOT CHANGE

% Standard spacing
\setlength{\parindent}{0ex} %DO NOT CHANGE
\setlength{\parskip}{1ex} %DO NOT CHANGE
\let\cleardoublepage\clearpage %DO NOT CHANGE


% Chapter title
\titleformat{\chapter}[display] %DO NOT CHANGE
{\Huge\bfseries}   % font size and style %DO NOT CHANGE
{\chaptername\ \thechapter} % label %DO NOT CHANGE
{20pt}             % spacing between label and title %DO NOT CHANGE
{\Huge}            % title formatting %DO NOT CHANGE

\titlespacing*{\chapter} %DO NOT CHANGE
{0pt}    % left margin %DO NOT CHANGE
{0pt}   % space before chapter title %DO NOT CHANGE
{20pt}   % space after chapter title %DO NOT CHANGE

% Section title %DO NOT CHANGE
\titleformat{\section} %DO NOT CHANGE
{\large\bfseries}  % font size and style %DO NOT CHANGE
{\thesection}      % label %DO NOT CHANGE
{1em}              % spacing between label and title %DO NOT CHANGE
{} %DO NOT CHANGE

\titlespacing*{\section} %DO NOT CHANGE
{0pt}    % left margin %DO NOT CHANGE
{10pt}   % space before section title %DO NOT CHANGE
{10pt}   % space after section title %DO NOT CHANGE

% Subsection title %DO NOT CHANGE
\titleformat{\subsection} %DO NOT CHANGE
{\normalsize\bfseries}  % font size and style %DO NOT CHANGE
{\thesubsection}      % label %DO NOT CHANGE
{1em}              % spacing between label and title %DO NOT CHANGE
{}

\titlespacing*{\subsection} %DO NOT CHANGE
{0pt}    % left margin %DO NOT CHANGE
{10pt}   % space before subsection title %DO NOT CHANGE
{0pt}   % space after subsection title %DO NOT CHANGE


%From here you can find useful commands to save time and make things look nice. You can use them, change them, or add to them at will

% Vectors, matrices etc.
\renewcommand{\v}[1]{\mathbf{#1}}      % Generic vector
\newcommand{\uv}[1]{\widehat{\mathbfit{#1}}}      % Generic unit vector
\renewcommand{\t}[1]{\mathsfbfit{#1}}	 % Generic tensor
\newcommand{\m}[1]{\mathsfbf{#1}}	 % Generic matrix
\newcommand{\tu}[1]{\bm{\mathsf{#1}}}	 % Upright tensor (for zero tensor)

%Vector operators
\renewcommand{\d}{{\mathrm d}}
\def \p {\partial}
\newcommand{\vnabla}{\bm{\nabla}}
\newcommand{\Lap}{{\Delta}} %Pure Laplacian
\newcommand{\aLap}{\nabla^2} %Applied Laplacian
%Hats, tildes and bars that fit
\renewcommand{\hat}{\widehat}
\renewcommand{\tilde}{\widetilde}
\renewcommand{\bar}{\overline}

% Differentiating 
\newcommand{\fd}[2]{\mathchoice{\frac{\d #1}{\d #2}}{\d #1/\d #2}{\d #1/\d #2}{\d #1/\d #2}}
\newcommand{\fdd}[2]{\mathchoice{\frac{\d^2 #1}{\d #2^2}}{\d^2 #1/\d #2^2}{\d^2 #1/\d #2^2}{\d^2 #1/\d #2^2}}
\newcommand{\fddd}[2]{\mathchoice{\frac{\d^3 #1}{\d #2^3}}{\d^3 #1/\d #2^3}{\d^3 #1/\d #2^3}{\d^3 #1/\d #2^3}}
\newcommand{\fdddd}[2]{\mathchoice{\frac{\d^4 #1}{\d #2^4}}{\d^4 #1/\d #2^4}{\d^4 #1/\d #2^4}{\d^4 #1/\d #2^4}}
\newcommand{\fdn}[3]{\mathchoice{\frac{\d^{#1} #2}{\d #3^{#1}}}{\d^{#1} #2/\d #3^{#1}}{\d^{#1} #2/\d #3^{#1}}{\d^{#1} #2/\d #3^{#1}}}
\newcommand{\pd}[2]{\mathchoice{\frac{\p #1}{\p #2}}{\p #1/\p #2}{\p #1/\p #2}{\p #1/\p #2}}
\newcommand{\pdd}[2]{\mathchoice{\frac{\p^2 #1}{\p #2^2}}{\p^2 #1/\p #2^2}{\p^2 #1/\p #2^2}{\p^2 #1/\p #2^2}}
\newcommand{\pddmixed}[3]{\mathchoice{\frac{\p^2 #1}{\p #2 \p #3}}{\p^2 #1 /\p #2 \p #3}{\p^2 #1 /\p #2 \p #3}{\p^2 #1 /\p #2 \p #3}}
\newcommand{\pddd}[2]{\frac{\p^3 #1}{\p #2^3}}
\newcommand{\pdn}[3]{\mathchoice{\frac{\p^{#1} #2}{\p #3^{#1}}}{\p^{#1} #2/\p #3^{#1}}{\p^{#1} #2/\p #3^{#1}}{\p^{#1} #2/\p #3^{#1}}}

% Misc
\DeclareMathSymbol{\Delta}{\mathalpha}{operators}{1} % Keeps Delta upright
\DeclareMathSymbol{\Gamma}{\mathalpha}{operators}{0} % Keeps Gamma upright
\newcommand{\ep}{\varepsilon} % A more normal epsilon

% Upright e and i for exponential and imaginary number
\newcommand{\e}{\mathrm{e}}
\renewcommand{\i}{\mathrm{i}}

	
\begin{document} %DO NOT REMOVE the title page. Just edit it.
\frontmatter
	% Pretty title page
	\thispagestyle{empty}
	\begin{titlepage}
		\begin{center}
			\includegraphics[scale=0.8]{images/durham-logo.pdf}
			\vspace{45mm}
			
			\Huge \textbf{All your nodes are belong to us: An exploration of percolation-like disease modelling.}
			
			\vspace{15mm}
			\large
			\begin{align*}
				\textit{by} &\quad \text{Matthew Evans} \\[4mm]
				\textit{Supervisor} &\quad \text{Dr Andrew Krause}
			\end{align*}
			\vspace{35mm}
			
			A dissertation submitted for the degree of \\
			\emph{MMath Mathematics}
			\vspace{5mm}
			
			\today
		\end{center}
	\end{titlepage}
	\normalsize
	
	% ----- PLAGIARISM DECLARATION ----- 
        % DO NOT REMOVE. WRITE YOUR NAME IN THE APPROPRIATE PLACE TO ACKNOWLEDGE THAT YOU HAVE READ, HAVE UNDERSTOOD, AND CONFIRM THAT YOU ADHERE TO THE STATEMENT.
	\begin{center}
		\large\textbf{Plagiarism declaration}
	\end{center}
	\begin{quote}
		This piece of work is a result of my own work and I have complied with the Department’s guidance on multiple submission and on the use of AI tools. Material from the work of others not involved in the project has been acknowledged, quotations and paraphrases suitably indicated, and all uses of AI tools have been declared.
	\end{quote}
	\vfill
	I, \underline{Matthew Evans} confirm that I have read, understood, and have adhered to the plagiarism declaration above.
	
	\newpage
	
		% ----- Declaration of use of AI tools in the writing process. ----- DO NOT REMOVE. Fill as needed 
	\begin{center}
		\large\textbf{Declaration of use of AI tools in the writing process}
	\end{center}
	\begin{center}
                No AI tools were used in this project, for any purpose.
	\end{center}
	
	
	\newpage
	
	% ----- ABSTRACT ----- % You may remove if you want or modify upon agreement with your supervisor
	\begin{center}
		\large\textbf{Abstract}
	\end{center}
	\begin{quote}
                TODO: write one of these
	\end{quote}
	
	\newpage
	
	% ----- ACKNOWLEDGEMENTS ----- % You may remove if you want or modify upon agreement with your supervisor

	% ----- LIST OF FIGURES ----- % You may remove if you want or modify upon agreement with your supervisor
	
	\listoffigures
	\markboth{}{} 
	%	The name of the figure that appears in the list is taken from the short description of the caption, i.e.
	% \caption[short caption - appears in the list of figures]{long caption - appears under the figure}
	% if no short description appears, i.e. if you use \caption{something} then 'something' will appear in the list of figures (and might be a very long sentence!).
	
	\newpage
	
	% ----- LIST OF TABLES----- % You may remove if you want or modify upon agreement with your supervisor
	
	\listoftables
	\markboth{}{} 
	% You will need to use the Table environment (not Tabular) and to use \caption in order for the table to appear in the above list (you can wrap the tabular environment within a table environemnt, see an example below). Much like with the list of figures, the short description of the caption is the one that is taken for the list of tables. 
	
	\newpage
	
	% ----- LIST OF NOTATION----- % You may remove if you want or modify upon agreement with your supervisor
	
	\chapter*{Notation}

	
	\begin{tabular}{p{4cm} l  }
	\end{tabular}
	
	
	% Pretty table of contents - DO NOT CHANGE OR REMOVE
	\setlength{\parskip}{0.2ex}
	\setcounter{tocdepth}{1}
	\renewcommand{\baselinestretch}{0.95}\normalsize
	\tableofcontents
	\renewcommand{\baselinestretch}{1}\normalsize
	\markboth{}{} 
	\setlength{\parskip}{1ex}
	

\mainmatter

\chapter{Introduction}

Throughout human history, disease has been one of the greatest societal challenges. From the black death, to COVID-19, widespread infection has disrupted society, caused economic recessions, and caused suffering to millions of people \citep{prabhuHistorysSevenDeadliest}. It is therefore no surprise that Epidemiology has a rich and vaired history, as we have tried to understand diseases from their chemical functions, to their effect on the human body, to how they spread through a population. It is this last problem which this report is concerned with.

Disease modelling has the potential to impact governmental policy with regards to pandemic preparedness and responses, as well as outbreak containment and response \citep{gov.ukEpidemiologicalModellingFrequently}. It is therefore of great importance to develop accurate and realistic models for disease spread, and for these to be flexible enough to be effective in their use \citep{lancetHowModellingCan2024}. It is also important to make these models easy to interpret and communicate about. The effectiveness of this was evident during the COVID-19 pandemic, as a great number of people were encouraged to ``flatten the curve'' - a direct reference to the underlying models used to predict the effect of the pandemic \citep{ecdcVideoCOVID19Flatten2020}.

Now the effects of the COVID-19 pandemic are waning, and so it is important to build upon the huge amount of work that was done surrounding it, to develop more accurate and useful models for other diseases with different attributes, such as passive immunity from birth. These models can help to shape public policy surrounding not just pandemics and outbreaks, but also vaccination programmes, and other forms of aid globally.

\chapter{Background}

Many approaches to disease modelling utilise a compartmental model, which in the years following the COVID-19 pandemic have become known to a much wider section of the general public than before. Many are therefore familiar with terms such as "R-value", as this was often reported in the media, for example \citet{bbcCoronavirusWhatNumber2020}. The models the public are familiar with are generally of the SIR type, where the population are grouped into three compartments: Susceptible, Infected, and Removed. However, this model is part of a larger family of compartmental models, the simplest being the SI model which does not have the Removed section. There are also more complex versions, such as the MSEIRS model, which models passive immunity, latency periods, and temporary immunity, which have existed for quite some time \citep{menonModellingSimulationCOVID192020}.

These models generally make an assumption of spatial homogeneity, and so more complex models have been developed to address this shorcoming. These include reaction-diffusion models \citep{davydovychReactionDiffusionEquations2023}, and models with interacing subpopulations \citep{keeling17MetapopulationDynamics2004}. However, these still make an assumption that the disease spread can be modelled in a continuous space; logically though, we know that disease spread is fundamentally a discretised process, as an individual is either infected or not.

There has therefore been some work put into stochastic simulations of these systems, in order to more accurately model the discrete nature of the populations \citep{stollenwerkMasterEquationSolution2000}. However, these also unfortunately lack the capacity to examine the effects of connections between individuals directly, and generally assume a well-mixed population.

To address this deficiency, a large amount of work has been done modelling diseases on various networks. Some of these are reminiscent of percolation theory results, such as giant component formation on random graphs \citep{browneInfectionPercolationDynamic2021}. Another recent approach modelled random walkers on random graphs, to mimic human movement patterns and model vector-born diseases \citep{grangerStochasticCompartmentModel2024}. In \citet{croccoloSpreadingInfectionsRandom2020}, the authors applied percolation-theoretic methods and results to a model of the COVID-19 pandemic in the USA, and explored the effects of lockdowns on disease spread.

I seek here to expand on this previous work, by creating a model of disease spread on a random network, and observing how changes both to the network structure and to the disease parameters qualitatively affect the epidemic. These results are compared to stochastic simulations based on a Chemical Master Equation approach to the population modelling, using the Gillespie SSA.

\chapter{The Chemical Master Equation Model}

\chapter{Creating the Model}

\chapter{Implementing the Model}

\chapter{Analysis of Criticality}

\chapter{Comparison between the networked and CME model}

\bibliography{mybib.bib}

\end{document}